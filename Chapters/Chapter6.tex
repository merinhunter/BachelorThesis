% Chapter 6: Last Conclusions

\chapter{Conclusiones y líneas de desarrollo futuras} % Main chapter title

\label{Chapter6} % Reference

%----------------------------------------------------------------------------------------

Al comienzo del proyecto se han establecido unos objetivos específicos a
cumplir:

\begin{itemize}
  \item Proporcionar un canal por el que transmitir la información.
  \item Desarrollar un esquema que permita cifrar y descifrar la información
  que queremos transmitir.
  \item Diseñar una forma de comprobar la integridad y la autenticación de la
  información.
\end{itemize}

Estos objetivos se han cumplido de manera más o menos satisfactoria: se ha
creado una vía para que distintos usuarios puedan intercambiar información,
estando ésta convenientemente cifrada y firmada.

Sin embargo, el proyecto tenía como objetivo principal crear una aplicación
accesible a cualquier usuario acostumbrado a un sistema de mensajería. Este
objetivo no se ha llegado a cumplir, ya que la aplicación resultante requiere
un manejo demasiado complicado para un usuario estándar.

Las líneas de desarrollo futuras se centran en mejorar la accesibilidad y la
interfaz de la aplicación.

En futuras iteraciones se podría añadir más funcionalidad al intercambio de
claves, como la generación de un código QR que permitiera a un usuario dar a
conocer su clave pública.

Algunos ajustes de personalización como poder modificar el tamaño de bloque de
cifrado, una interfaz más vistosa o un sistema de notificaciones para cuando,
por ejemplo, recibes un mensaje harían que un mayor número de usuarios usara
la aplicación.

Por último, un cambio que haría más sencillo el uso de la aplicación sería
modificar la manera en la que se suben los fragmentos al servidor (actualmente
se hace mediante scp) por un sistema que permitiera hacer llamadas POST.

Todo el código utilizado en este TFG está subido en un repositorio de
GitHub\footnote{\url{https://github.com/merinhunter/Shatter}}.
