% Chapter 6: Last Conclusions

\chapter{Conclusiones y líneas de desarrollo futuras} % Main chapter title

\label{Chapter6} % Reference

%-------------------------------------------------------------------------------

Los objetivos establecidos al comienzo del proyecto son:

\begin{itemize}
  \item Proporcionar un canal por el que transmitir la información.
  \item Desarrollar un esquema que permita cifrar y descifrar la información que se quiere transmitir.
  \item Diseñar una forma de comprobar la integridad y la autenticación de la información.
\end{itemize}

Estos objetivos se han cumplido de manera más o menos satisfactoria: se ha creado una vía para que distintos usuarios puedan intercambiar información, estando esta convenientemente cifrada y firmada.

Sin embargo, el objetivo principal del proyecto es crear una aplicación accesible a cualquier usuario acostumbrado a un sistema de mensajería. Este objetivo no se ha cumplido, ya que la aplicación resultante requiere un manejo demasiado complicado para un usuario estándar.

Las líneas de desarrollo futuras se centran en mejorar la accesibilidad y la interfaz de la aplicación.

Estas nuevas iteraciones se basan en añadir más funcionalidad al intercambio de claves, como la generación de un código QR que permite a un usuario dar a conocer su clave pública.

Algunos ajustes de personalización como poder modificar el tamaño de bloque de cifrado, una interfaz más vistosa o un sistema de notificaciones son otras mejoras que añadir en siguientes versiones.

Por último, un cambio que hace más sencillo el uso de la aplicación es modificar la manera en la que se suben los fragmentos al servidor (actualmente se hace mediante scp) por un sistema que permite hacer llamadas POST.

Todo el código utilizado en este TFG está disponible en un repositorio de GitHub\footnote{\url{https://github.com/merinhunter/Shatter}}.
