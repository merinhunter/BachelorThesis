% Chapter 1: Introduction

\chapter{Introducción} % Main chapter title

\label{Chapter1} % For referencing this chapter elsewhere, use \ref{Chapter2}

%----------------------------------------------------------------------------------------

\section{Motivación}

Las comunicaciones seguras nacen del deseo de protegernos: de proteger con quién nos comunicamos y el qué comunicamos.

De este deseo surgen multitud de protocolos de seguridad que hoy en día usamos sin darnos cuenta. Desde una simple consulta web hasta la felicitación de Año Nuevo, nuestras comunicaciones pasan por diversas operaciones para preservar su seguridad.

Esta seguridad viene generalmente proporcionada por la confianza que depositamos en ciertas organizaciones, las cuales crean una red de confianza sobre la que se sustenta todo este sistema. Pero, ¿qué sucede si no podemos confiar en estas entidades? ¿Y si queremos ser nosotros los responsables de proporcionar la seguridad?

Con esta idea comienza la búsqueda de una herramienta que permita a los usuarios ser los artífices de su propia red de confianza, el pilar central en seguridad.

%----------------------------------------------------------------------------------------

\section{Estructura de la memoria}
