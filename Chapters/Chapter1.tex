% Chapter 1: Introduction

\chapter{Introducción} % Main chapter title

\label{Chapter1}

%-------------------------------------------------------------------------------

% Define some commands to keep the formatting separated from the content
\newcommand{\keyword}[1]{\textbf{#1}}
\newcommand{\tabhead}[1]{\textbf{#1}}
\newcommand{\code}[1]{\texttt{#1}}
\newcommand{\file}[1]{\texttt{\bfseries#1}}
\newcommand{\option}[1]{\texttt{\itshape#1}}
\newcommand{\Mod}[1]{\ (\mathrm{mod}\ #1)}

%-------------------------------------------------------------------------------

\section{Motivación}

Las comunicaciones seguras nacen del deseo de protegernos: de proteger con quién nos comunicamos y qué comunicamos.

De este deseo surgen multitud de algoritmos de seguridad que hoy en día usamos sin darnos cuenta. Desde una simple consulta web hasta la felicitación de Año Nuevo, nuestras comunicaciones pasan por diversas operaciones para preservar su seguridad.

Esta seguridad viene generalmente proporcionada por la confianza que depositamos en ciertas organizaciones, entidades que crean una red de confianza sobre la que se sustenta todo este sistema.\footnote{Nos referimos a las Autoridades Certificadoras.} Pero, ¿qué sucede si de quién nos queremos proteger es de ellos? ¿Por qué tenemos que confiar en que una entidad gubernamental sea la que preserve la seguridad de nuestras comunicaciones? ¿Por qué no podemos tener nuestra propia red de confianza?

Con esta idea comienza el desarrollo de una herramienta que nos permita mantener comunicaciones seguras sin depender de una Autoridad Certificadora para ello.

%-------------------------------------------------------------------------------

\section{Estructura de la memoria}

La estructura que se va a seguir en este proyecto es la siguiente:

\begin{itemize}
  \item En el Capítulo~\ref{Chapter2} se presentan los objetivos que se persiguen con este proyecto y un modelo de amenaza para el mismo.
  \item En el Capítulo~\ref{Chapter3} se explican varios conceptos y tecnologías que ya existen y que se han usado para llevar a cabo el proyecto.
  \item En el Capítulo~\ref{Chapter4} se detallan los diferentes prototipos y la arquitectura de seguridad de la aplicación, así como el tiempo dedicado a cada prototipo.
  \item En el Capítulo~\ref{Chapter5} se exponen los resultados obtenidos con ejemplos y los problemas encontrados durante el desarrollo del proyecto.
  \item En el Capítulo~\ref{Chapter6} se finaliza la memoria haciendo una reflexión sobre los resultados y las posibles líneas de desarrollo futuras.
\end{itemize}
