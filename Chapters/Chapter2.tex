% Chapter 2: Objectives

\chapter{Objetivos} % Main chapter title

\label{Chapter2}

%----------------------------------------------------------------------------------------

\section{Objetivo general}

El objetivo principal es el desarrollo de una aplicación para el sistema operativo
Android, que permita a cualquier usuario realizar una comunicación de datos
de manera que se preserve la \keyword{confidencialidad}, la \keyword{integridad}
y la \keyword{autenticación} de la información transmitida.

%----------------------------------------------------------------------------------------

\section{Objetivos específicos}

Para abordar el objetivo principal del proyecto, éste se ha dividido en unos
objetivos más específicos:

\begin{itemize}
  \item Proporcionar un canal por el que transmitir la información.
  \item Desarrollar un esquema que permita cifrar y descifrar la información que queremos transmitir.
  \item Diseñar una forma de comprobar la integridad y la autenticación de la información.
\end{itemize}

%----------------------------------------------------------------------------------------

\section{Modelo de amenaza}

En criptografía, un modelo de amenaza (\emph{threat model}) especifica que tipo
de amenazas potenciales a un sistema puede realizar un individuo (o grupo) con
unos determinados recursos. En base a estas amenazas, se desarrolla un
determinado esquema de seguridad para poder contrarrestarlas.

Nuestro modelo de amenaza será un atacante con conocimientos de criptoanálisis,
utilizando para el ataque una máquina con un procesamiento computacional estándar.

Los puntos más vulnerables de nuestro modelo son el hardware y el software, por
lo que dejamos fuera a aquellas compañías que han trabajado en su desarrollo,
como Intel, Google, Qualcomm, etc.

Debido a los recursos que poseen también dejamos fuera de nuestro modelo a
cualquiera de los gobiernos de las grandes potencias del mundo, como China, Corea,
Rusia o Estados Unidos.

Un atacante de nuestro modelo de amenaza podría realizar los siguientes tipos de ataque:

\begin{itemize}
  \item Ataques sobre el texto cifrado -- En este tipo de ataques el atacante
  dispone únicamente del texto cifrado y no tiene acceso al texto plano.
  Ejemplos de este modelo son los ataques por fuerza bruta, en los que se prueba
  cada una de las posibles combinaciones para una clave hasta dar con la correcta.

  \item Ataque de texto plano conocido -- En este tipo de ataques se presupone
  que el atacante tiene acceso a un número limitado de textos planos y sus
  correspondientes textos cifrados. El objetivo de este tipo de ataques es el de
  obtener la clave de cifrado a partir de estos pares, con el fin de poder
  desencriptar futuras comunicaciones.

  \item Ataque de texto plano selectivo -- En este tipo de ataques el atacante puede
  seleccionar un número indeterminado de textos planos y obtener sus equivalentes
  cifrados. Con esto, un atacante puede probar distintas combinaciones y encontrar
  patrones sobre el texto plano para explotar alguna vulnerabilidad.

  \item Ataque de texto cifrado selectivo -- Parecido al anterior, el atacante
  puede obtener un número indeterminado de textos planos a partir de sus
  equivalentes cifrados.

  \item Ataques sobre la clave -- En este tipo de ataques el atacante dispone de
  alguna información acerca de la clave de cifrado.

  \item Man in the middle

  \item Ataques de side channel

  \item Ataques de reply
\end{itemize} \emph{\parencite{Reference20}}\\
