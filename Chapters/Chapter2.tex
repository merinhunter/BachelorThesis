% Chapter 2: Objectives

\chapter{Objetivos} % Main chapter title

\label{Chapter2}

%----------------------------------------------------------------------------------------

\section{Objetivo general}

El objetivo principal es el desarrollo de una aplicación para el sistema operativo
Android, que permita a cualquier usuario realizar una comunicación de datos
de manera que se preserve la \keyword{confidencialidad}, la \keyword{integridad}
y la \keyword{autenticación} de la información transmitida.

%----------------------------------------------------------------------------------------

\section{Objetivos específicos}

Para abordar el objetivo principal del proyecto, éste se ha dividido en unos
objetivos más específicos:

\begin{itemize}
  \item Proporcionar un canal por el que transmitir la información.
  \item Desarrollar un esquema que permita encriptar y desencriptar la información que queremos transmitir.
  \item Diseñar una forma de comprobar la integridad y la autenticación de la información.
\end{itemize}

%----------------------------------------------------------------------------------------

\section{Modelo de amenaza}

En criptografía, un modelo de amenaza (\emph{threat model}) especifica que tipo
de amenazas potenciales a un sistema puede realizar un individuo (o grupo) con
unos determinados recursos.

Para nuestro caso vamos a poner como modelo de amenaza a un usuario con algunos
conocimientos de criptoanálisis, utilizando para el ataque una máquina con un
procesamiento computacional estándar.\footnote{De nuestro modelo de amenaza
quedan fuera aquellos en los que se presupone que quien nos ataca es una compañía
tecnológica del tipo Intel, Google, Qualcomm o algún gobierno importante como el
de China, Corea, Rusia o Estados Unidos.}

Un usuario de este tipo podría realizar los siguientes tipos de ataque:

\begin{itemize}
  \item Ataques sobre el texto cifrado -- En este tipo de ataques el atacante
  dispone únicamente del texto cifrado y no tiene acceso al texto plano.
  Ejemplos de este modelo son los ataques por fuerza bruta, en los que se prueba
  cada una de las posibles combinaciones para una clave hasta dar con la correcta.

  \item Ataque de texto plano conocido -- En este tipo de ataques se presupone
  que el atacante tiene acceso a un número limitado de textos planos y sus
  correspondientes textos cifrados. El objetivo de este tipo de ataques es el de
  obtener la clave de cifrado a partir de estos pares, con el fin de poder
  desencriptar futuras comunicaciones.

  \item Ataque de texto plano selectivo -- En este tipo de ataques el atacante puede
  seleccionar un número indeterminado de textos planos y obtener sus equivalentes
  cifrados. Con esto, un atacante puede probar distintas combinaciones y encontrar
  patrones sobre el texto plano para explotar alguna vulnerabilidad.

  \item Ataque de texto cifrado selectivo -- Parecido al anterior, el atacante
  puede obtener un número indeterminado de textos planos a partir de sus
  equivalentes cifrados.

  \item Ataques sobre la clave -- En este tipo de ataques el atacante dispone de
  alguna información acerca de la clave de cifrado.
\end{itemize} \emph{\parencite{Reference20}}\\
