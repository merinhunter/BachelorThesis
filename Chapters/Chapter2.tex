% Chapter 2: Objectives

\chapter{Objetivos} % Main chapter title

\label{Chapter2}

%----------------------------------------------------------------------------------------

\section{Objetivo general}

El objetivo principal es el desarrollo de una aplicación para el sistema operativo
Android, que permita a cualquier usuario realizar una comunicación de datos
de manera que se preserve la \keyword{confidencialidad}, la \keyword{integridad}
y la \keyword{autenticación} de la información transmitida.

%----------------------------------------------------------------------------------------

\section{Objetivos específicos}

Para abordar el objetivo principal del proyecto, éste se ha dividido en unos
objetivos más específicos:

\begin{itemize}
  \item Proporcionar un canal por el que transmitir la información.
  \item Desarrollar un esquema que permita encriptar y desencriptar la información que queremos transmitir.
  \item Diseñar una forma de comprobar la integridad y la autenticación de la información.
\end{itemize}

%----------------------------------------------------------------------------------------

\section{Modelo de atacante}

En criptografía, un modelo de atacante especifica que recursos y que tipo de acceso
tiene un criptoanalista a un sistema cuando intenta \emph{romper} un mensaje cifrado
generado por este mismo sistema.

Existen varios modelos de atacante:

\begin{itemize}
  \item Ataques sobre el texto cifrado -- En este modelo el atacante dispone únicamente
  del texto cifrado y no tiene acceso al texto plano. Este modelo es el más débil de todos,
  ya que para que un atacante suponga un riesgo, debe conocer algo sobre el texto plano.
  Ejemplos de este modelo son los ataques por fuerza bruta, en los que se prueba
  cada una de las posibles combinaciones para una clave hasta dar con la correcta.

  \item Ataque de texto plano conocido -- En este tipo de ataques se presupone
  que el atacante tiene acceso a un número limitado de textos planos y sus
  correspondientes textos cifrados. El objetivo de este tipo de ataques es el de
  obtener la clave de cifrado a partir de estos pares, con el fin de poder
  desencriptar futuras comunicaciones.

  \item Ataque de texto plano selectivo -- En este modelo el atacante puede
  seleccionar un número indeterminado de textos planos y obtener sus equivalentes
  cifrados. Con esto, un atacante puede probar distintas combinaciones y encontrar
  patrones sobre el texto plano para explotar alguna vulnerabilidad.

  \item Ataque de texto cifrado selectivo -- Parecido al anterior, el atacante
  puede obtener un número indeterminado de textos planos a partir de sus
  equivalentes cifrados.

  \item Ataques sobre la clave -- En este tipo de ataques el atacante dispone de
  alguna información acerca de la clave de cifrado.

  \item Ataque de canal lateral -- Este modelo de ataque se basa en usar
  información especial sobre el proceso de encriptación y desencriptación de un
  mensaje con el fin de obtener algo de información de él. Se utilizan recursos
  como el ruido eléctrico que genera una operación de cifrado o el tiempo que
  toma en ejecutarse.
\end{itemize} \emph{\parencite{Reference20}}\\

Todos estos modelos descritos serán más o menos efectivos en función de los
recursos de los que disponga el atacante. Con el proyecto se pretende hacer
frente a todos ellos para un atacante estándar.

Si un atacante con unos recursos equiparables a los de una nación intentara
romper las medidas de seguridad implementadas, es posible que no se puediera
garantizar el objetivo principal que persigue este proyecto.
