% Chapter 5: Results

\chapter{Resultados} % Main chapter title

\label{Chapter5} % Reference

%----------------------------------------------------------------------------------------

\section{Ejemplos de utilidad}

%----------------------------------------------------------------------------------------

\section{Problemas encontrados}

A lo largo de la etapa que supuso el desarrollo de la aplicación, me encontré
con varios problemas que dificultaron la finalización del proyecto.

El mayor problema con el que se tuvo que lidiar fue el de no haber creado una
parte portable que hiciera más fácil la migración a Android. Algunas librerías
usadas cuando el proyecto solo era una aplicación escrita en Java tenían ciertas
dependencias, las cuales no se encontraban en ninguna librería Java usada en
Android. Además, el uso del KeyStore de Android para almacenar las claves
exigía realizar el cifrado asimétrico, las firmas y la generación de las claves
de una manera determinada, lo que supuso la eliminación de las clases que se
encargaban anteriormente de ello (RSALibrary y RSAPSS). Aunque algunos de estos
problemas no se habrían podido solucionar aun teniendo una parte portable, sin
duda habría supuesto un ahorro considerable de tiempo.

A raíz de lo anterior, la mayoría de las clases no estuvieron bien definidas
desde un principio. Esto supuso muchos cambios a lo largo del desarrollo de la
aplicación (Cabeceras, modos de cifrado, registros...). De nuevo, una buena
planificación habría ahorrado una gran cantidad de tiempo y energía.

Pero no todos los problemas encontrados tuvieron que ver con la desorganización.
La aplicación ahora cuenta con un servidor dedicado pero, en un principio, se
pensó en utilizar un servicio externo para el almacenamiento de los mensajes
enviados por los usuarios. De varias ideas que salieron, se eligió la aplicación
Pastebin\footnote{\url{https://pastebin.com/}}, ya que permite la subida anónima de
texto plano. La idea era que la aplicación subiera los distintos fragmentos
cifrados a la plataforma, permitiendo al resto de usuarios su descarga. Sin
embargo, la existencia de límites en el número de mensajes enviados o en la
longitud de los mismos hicieron que se desechara la idea en favor de un
servidor dedicado.
